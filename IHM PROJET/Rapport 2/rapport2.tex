\documentclass[12pt, a4paper]{article}

\usepackage[utf8]{inputenc}
\usepackage[T1]{fontenc}
\usepackage[francais]{babel}
\usepackage[top=2cm, bottom=2cm, left=2cm, right=2cm]{geometry}
\usepackage{verbatim}
\usepackage{graphicx}
\usepackage{listings}
\lstset{
%upquote=true,
columns=flexible,
basicstyle=\ttfamily,}

\title{Projet IHM - Rapport 2 - RICM4}
\author{\bsc{Fréby} Rodolphe - \bsc{Barbier} Jérome - \bsc{Husson} Augustin - \bsc{Labat} Paul}
\date{\today}



\begin{document}
\maketitle
\tableofcontents
\newpage

\section{Introduction}
Dans ce second rapport, nous allons désormais rendre compte de l'activité que devra permettre notre prototype, au travers de scénarios d'usages, modèle de tâches.

\section{Scénarios d'usages}
\subsection{Scénario 1 - Bob, un utilisateur novice}


Bob est un utilisateur novice, et il décide d'utiliser CTTE pour la première fois. Au lancement du logiciel, une fenêtre pop-up s'affiche lui demandant si il désire suivre un tutoriel pas à pas pour apprendre à manier le logiciel. Cependant, il ne possède pas le temps nécessaire, donc il décide de la fermer pour immédiatement tenter de réaliser ce qu'il souhaite. Il va donc tenter de comprendre rapidement l'utilisation des différentes fonctionnalités du logiciel, en passant sa souris au travers des menus et sur les boutons pour obtenir des informations.\\ 


Par la suite, il décide de cliquer sur le nœud déjà présent dans la zone de travail, afin de voir comment réagit le logiciel. Il remarque alors que la zone de travail situé sous les boutons principaux de gestion se met à jour, lui donnant une liste de champs définissant le nœud en question. Il décide donc de modifier les champs à sa convenance, afin de mettre à jour les données pour qu'elles correspondent à ses désirs.\\


Bob explore par la suite le dernier menu sur la gauche, afin d'obtenir des informations sur les icônes présentes. De suite, il remarque qu'il peut créer de nouveau nœud, et clique sur l'un d'eux, celui qui correspond au mieux à ce qu'il recherche d'après les infobulles. Le nouveau nœud ainsi créé se place en dessous du précédent. Il le sélectionne et met à jour les informations comme pour le nœud précédent. \\


Mais il a toujours besoin de créer de nouveaux nœuds, il va donc répéter l'opération, et remarquer que le nouveau nœud se place à la suite de celui qui est sélectionner dans la zone de travail. Il décide donc de le supprimer car celui-ci ne se situe pas là où il le désire.\\


Après quelques minutes de travail, son téléphone sonne, et on lui demande de se déplacer pour aller chercher sa fille au cinéma. Il décide donc de sauvegarder son travail avant de partir, pour cela il décide de cliquer sur le menu \emph{File}, puis \emph{Save as}. Une fois la sauvegarde terminé, il ferme le logiciel, celui lui demande si il veut sauvegarder avant de quitter, Bob choisi non, et s'absente.\\

\subsection{Scénario 2 - Alice, une utilisatrice experte}


Alice arrive au travail à 8h15, légèrement en retard, elle décide donc de reprendre immédiatement ce qu'elle a commencé hier. Elle lance CTTE, et une fois lancé, fait la commande \emph{CTRL + o} afin d'ouvrir son document. Elle rajoute des tâches et les modifies à sa convenance, en effectuant des sauvegardes rapides par \emph{CTRL + s}. Elle décide également de faire une sauvegarde en temps que \emph{JPG}, pour cela elle clique sur \emph{File}, puis \emph{Save as}, puis \emph{Save as JPG} \\


Ses collègues de travail l'appel pour un café, elle décide de fermer le logiciel le temps de son absence, pour cela elle effectue la commande \emph{CTRL + q}. Le logiciel lui signal qu'elle va quitter, et lui demande si elle désire sauvegarder son travail. Elle clique sur non, puis s'absente.\\

\subsection{Scénario 3 - Eve, une utilisatrice familière avec le logiciel}


Eve décide de modifier un document existant. Pour cela, elle clique sur le bouton \emph{open} afin de pouvoir charger son document. Une fois le document chargé, elle le modifie à sa guise, en cliquant sur le bouton propriété. Elle décide de le sauvegarder en cliquant sur l'icône correspondante.\\


Cependant, elle ne se rappel plus d'une manipulation particulière. Elle décide de cliquer sur \emph{Help} et d'ouvrir le menu de recherche dans l'aide. Une fois cela effectué, elle décide d’effectuer une recherche sur un nœud en particulier. Pour cela, elle utilise la fonction recherche en bas à gauche de l'interface. Elle décide par la suite de zoomer sur le résultat de sa recherche, à l'aide de l'icône correspondante. Elle clique de nouveau sur propriété, et modifie ce qu'elle souhaite. Elle clique ensuite sur la croix rouge afin de fermer le logiciel. Elle se rend compte qu'elle a oublié de le sauvegardé, grâce au message fourni par le logiciel. Elle décide de cliquer sur oui, puis sauvegarde son document avant de fermer le logiciel.

\end{document}